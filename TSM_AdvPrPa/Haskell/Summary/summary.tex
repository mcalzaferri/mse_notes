\documentclass[12pt]{article}
\usepackage{minted}
\begin{document}
\section*{Haskell}
\subsection*{GHCI commands}
\begin{minted}{haskell}
:load <name>
:reload
:set editor <name>
:edit <name>
:type <expr>
:? --show all commands
:quit
:set +t --outputs type information for everything
\end{minted}

\subsection*{Function definitions}
f a b + c * b means f(a,b) + c*b\\
function has higher priority then operators! f a + b means f(a) + b and not f(a+b)!\\
f (g x) means f(g(x))\\
f x y == x `f` y (Only with back quotes)\\
mult x y z means ((mult x)y)z when curried (Orther of function generation)\\
use parenthesis to transform 1 + 1 = (+) 1 1 or even (1+) 1 = (+1) 1 these are called "sections"\\

\subsection*{Writing functions}
functions start with a lower-case letter\\
list arguments have an s suffix for example xs, ns or others\\
In a sequence of definitions, each definition must begin in precisely the same column\\
\begin{minted}{haskell}
a = b + c
    where
      b = 1
      c = 2
--means
a = b + c where {b = 1;c = 2}
\end{minted}
a function is polymorphic if its type contains one or more type variables 
\begin{minted}{haskell}
length :: [a] -> Int
\end{minted}
a function is overloaded if its type contains one or more class constraints 
\begin{minted}{haskell}
(+) :: Num a => a -> a -> a
\end{minted}

\subsubsection*{Conditional Expression}
\begin{minted}{haskell}
signum n = if n < 0 then -1 else 
    if n == 0 then 0 else 1
\end{minted}

\subsubsection*{Guareded equations}
\begin{minted}{haskell}
signum n | n < 0     = -1
         | n == 0    = 0
         | otherwise = 1
\end{minted}

\subsubsection*{Pattern matching}
\begin{minted}{haskell}
True  && b = b
False && _ = False
--x:xs pattern:
head (x:_) = x --You MUST use parenthesis here
\end{minted}

\subsubsection*{Where clause}
\begin{minted}{haskell}
odds n = map f [0..n-1]
    where
        f x = x*2 + 1
\end{minted}

\subsubsection*{Lambdas}
\begin{minted}{haskell}
odds n = map (\x = x*2 + 1) [0..n-1]
\end{minted}

\subsubsection*{List comprehensions}
List comprehensions use generators (A generator is a list or a function returning a list)
\begin{minted}{haskell}
[(x,y) | x <- [1..3], y <- [x..3]] --[(1,1),(1,2),(1,3),(2,2),(2,3),(3,3)]
concat xss = [x | xs <- xss, x <- xs] -- xss is a list of lists
allPositiveInt = [0..] --generator without upper bound
--list comprehensions can use guards
[x | x < [1..10], even x]
factors n = [x | x <- [1..n], n `mod` x == 0] --returns all factors of a number
\end{minted}

\subsubsection*{Recursive functions}
\begin{minted}{haskell}
fac 0 = 1
fac n = n * fac (n-1)

length [] = 0
length (_:xs) = 1 + length xs 
--_ is used as the value of the element is not required

(++) :: [a] -> [a] -> [a]
[] ++ ys = ys
(x:xs) ++ ys = x : (xs ++ ys)

qsort :: Ord a => [a] -> [a]
qsort [] = []
qsort (x:xs) =
  qsort smaller ++ [x] qsort larger
  where
    smaller = [a | a <- xs, a <= x]
    larger = [b | b <- xs, b > x]
\end{minted}

\subsubsection*{Higher order functions}
Higher order functions are function with functions as argument or which return a function
\begin{minted}{haskell}
--foldr is a very important higher order function
--A number of functions on lists can be defined 
--using the following simple pattern of recursion:
f [] = value
f (x:xs) = x § f xs
--f maps the empty list to some value v,
--and any non-empty list to some function §
--applied to its head and f of its tail.

foldr :: Foldable t => (a -> b -> b) -> b -> t a -> b
foldr f v [] = v
foldr f v (x:xs) = f x (foldr f v xs)  
--foldr does just that applying the function recursively

reverse = foldr (\x xs -> xs ++ [x]) []

foldl :: Foldable t => (b -> a -> b) -> b -> t a -> b
--It takes the second argument and the first item of the list 
--and applies the function to them, 
--then feeds the function with this result and the second argument and so on. 
\end{minted}




\subsection*{Type declarations}
define types using existing types
\begin{minted}{haskell}
type String = [Char]
type Pos = (Int,Int)
type Trans = Pos -> Pos
type Tree = (Int,[Tree]) --!!!this will not work as it is recursive!!!
type Pair a = (a,a) --With type parameter. 
--Then we can define mult :: Pair Int -> Int
\end{minted}



\subsection*{Data declarations}
define completely new types with specific values
\begin{minted}{haskell}
data Bool = False | True --The values False/True are called "contructors"
data Nat = Zero | Succ Nat --recursive is allowed when creating new types
data Expr = Val Int 
          | Add Expr Expr --We can use recursion here
          | Mul Expr Expr

data Shape = Circle Float
           | Rect Float Float --Data declerations can have parameters as well
--Circle and Rect can be viewed as functions that contruct values of type Shape
Circle :: Float -> Shape
Rect :: Float -> Float -> Shape
--Now we can define
area :: Shape -> Float
area (Circle r) = pi * r^2
area (Rect x y) = x * y
--data declarations can also have parameters just like types
data Maybe a = Nothing | Just a
--Now we can define
safediv :: Int -> Int -> Maybe Int
safediv _ 0 = Nothing
safediv m n = Just (m `div` n)
\end{minted}



\subsection*{Lists and strings}
\begin{minted}{haskell}
(!!) :: [a] -> Int -> a --Returns element at nth position. Ex: [3,4,5] !! 2 == 4
(++) :: [a] -> [a] -> [a] --append
(:) :: a -> [a] -> [a] --called "cons" adds an element to the start of a list
head :: [a] -> a
last :: [a] -> a
tail :: [a] -> [a]
take :: Int -> [a] -> [a] --Get first n elements
drop :: Int -> [a] -> [a] --Get list without first n elements
length :: Foldable t => t a -> Int
sum_product :: (Foldable t, Num a) => t a -> a
reverse :: [a] -> [a]
init :: [a] -> a --Removes the last element
zip :: [a] -> [b] -> [(a,b)] --pair element of [a] with the corresponding of [b]
fst :: (a,b) -> a --return the first element of a tuple of type (a,b)
null :: Foldable t => t a -> Bool --returns true if the list is empty
concat :: [[a]] -> [a]
replicate :: Int -> a -> [a] --creates a list with n identical elements
elem :: (Foldable t, Eq a) => a -> t a -> Bool --Basically "contains" 
map :: (a -> b) -> [a] -> [b]
\end{minted}


\subsection*{Important functions}
\begin{minted}{haskell}
(+)_(-)_(*) :: Num a => a -> a -> a
(/) :: Fractional a => a -> a -> a
div_mod :: Integral a => a -> a -> a
(==)_(/=) :: Eq a => a -> a -> Bool
(<)_(>)_(>=)_(<=) :: Ord a => a -> a -> Bool --Ord includes Eq
(.) :: (b -> c) -> (a -> b) -> (a -> c) --composition of 2 functions
abs :: Num a => a -> a
even_odd :: Integral a => a -> Bool
and_or :: Foldable t => t Bool -> Bool
filter :: (a -> Bool) -> [a] -> [a]
all_any :: Foldable t => (a -> Bool) -> t a -> Bool --Do all satisfy a predicate
takeWhile_dropWhile :: (a -> Bool) -> [a] -> [a] --Take elements until false
\end{minted}

\subsection*{Types}
Applying a function to one or more arguments of the wrong type is called a type error.\\
If evaluating an expression e would produce a value of type t, then e has type t, written
\begin{minted}{haskell}
e :: t
\end{minted}
Every well formed expression has a type, which can be automatically calculated at compile time using a process called type inference.\\
All type errors are found at compile time, which makes programs safer and faster by removing the need for type checks at run time.\\
Basic types: Bool, Char, String, Int, Float\\
A tuple is a sequence of values of different types
\begin{minted}{haskell}
(False,'a',True) :: (Bool,Char,Bool)
\end{minted}
A function is a mapping from values of one type to values of another type\\
Functions are curried by default, meaning they return another function if more then 1 argument is given\\

\end{document}